% This must be in the first 5 lines to tell arXiv to use pdfLaTeX, which is strongly recommended.
\pdfoutput=1
% In particular, the hyperref package requires pdfLaTeX in order to break URLs across lines.

\documentclass[11pt]{article}

% Change "review" to "final" to generate the final (sometimes called camera-ready) version.
% Change to "preprint" to generate a non-anonymous version with page numbers.
\usepackage[preprint]{acl}

% Standard package includes
\usepackage{times}
\usepackage{latexsym}

% For proper rendering and hyphenation of words containing Latin characters (including in bib files)
\usepackage[T1]{fontenc}
% For Vietnamese characters
% \usepackage[T5]{fontenc}
% See https://www.latex-project.org/help/documentation/encguide.pdf for other character sets

% This assumes your files are encoded as UTF8
\usepackage[utf8]{inputenc}

% This is not strictly necessary, and may be commented out,
% but it will improve the layout of the manuscript,
% and will typically save some space.
\usepackage{microtype}

% This is also not strictly necessary, and may be commented out.
% However, it will improve the aesthetics of text in
% the typewriter font.
\usepackage{inconsolata}

% If the title and author information does not fit in the area allocated, uncomment the following
%
%\setlength\titlebox{<dim>}
%
% and set <dim> to something 5cm or larger.

\usepackage{natbib}

\title{Attacks on Neural Networks in a Lightweight Speech Anonymization Pipeline}

\author{Daan Brugmans \\
  Radboud University\\
  \texttt{daan.brugmans@ru.nl}
}

\begin{document}

\maketitle

\begin{abstract}
  
\end{abstract}

\section{Introduction}
As advancements in the field of Automatic Speech Recognition (ASR) have accelerated with the rise of modern End-to-End neural networks, the risks associated with using such models in ASR applications has become more evident.

Modern neural ASR models are capable of parsing and producing speech to a new level of authenticity: transformers are State-of-the-Art for ASR and word recognition, and the introduction of modern unsupervised deep neural network architectures, such as Generative Adversarial Networks (GANs) and Variational Autoencoders (VAEs), has allowed for more realistic, accurate, and easier generation of speech.
These modern speech generation models are capable of learning to reproduce a person's voice, and then generating new speech using the learned voice.
Such synthesized speeches are called \textit{deepfaked} speeches, or simply \textit{deepfakes}.

The presence and influence of deepfakes has become increasingly apparent in recent years: neurally synthesized audio and video of important persons are used to spread misinformation and manipulate.
One way to counteract the repercussions of deepfakes is the removal of the personalization in the learned, and thus reproduced, speech.
This is called \textit{Speaker Anonymization}.
In Speaker Anonymization, we aim to maintain the ASR quality of the audio, while applying changes that make the audio untraceable to a person's likeness.

Although modern Speaker Anonymization systems, often neural in nature, have been shown to be able to anonymize speech while maintaining ASR quality, they can also be manipulated.
By attacking neural Speaker Anonymization systems, we may be able to circumvent the preventative measures they provide, and generate speech to a person's likeness regardless of their presence.
This paper will focus in that topic: attacking neural Speaker Anonymization systems.

\section{Related Work}
\citet{meyer2023anonymizing} % Work on neural speaker anonymization

\citet{shihao2024adversarial} % Work on attacking neural speaker anonymization

\citet{xiaoyong2019adversarial} % Often cited paper about adversarial examples and attacks, use for introducing the idea of attacks on neural networks

\citet{goodfellow2015explaining} % The paper about FGSM, use as intro to evasion attacks

\citet{madry2018towards} % The paper about PGD, which we will use, builds upon FGSM

\citet{gu2019badnets} % The paper about BadNets. Keep it brief. Use it to visualize the backdooring process.

\citet{liu2018trojaning} % The paper about trojanning audio. Keep it brief. Use it to show that backdooring audio is also possible.

\citet{stefanos2022ultrasonic} % The paper about ultrasonic frequencies for backdooring. Keep it brief. Another example of audio backdoor.

\citet{stefanos2023jingleback} % Jingleback paper. Explain in more details since you will focus on it.

\section{Method}
\citet{kai2022lightweight} % Work on lightweight speaker anonymization, idea for our method/pipeline.

\section{Experiment}

\section{Results}

\section{Discussion}

\section{Conclusion}

\bibliography{custom}

\appendix

\end{document}